The State of the Art, also known as the Literature Review (or Foundations), serves a cluster of very important aims. First of all, it demonstrates that you have built a solid knowledge of the field where the research is taking place, that you are familiar with the main issues at stake, and that you have critically identified and evaluated the key
literature. On the other hand, it shows that you have created an innovative and coherent view integrating and synthesising the main aspects of the field, so that you can now put into perspective the new direction that you propose to explore. The State of the Art must give credit to the authors who laid the groundwork for your research, so that when, in the following chapter, your research objectives are further clarified, the reader is able to recognise beyond doubt that what you are attempting to do has not been done in the past and that your research will likely make a significant contribution to the literature.

Example of single equations \eqref{eq_1} with equation numbers and reference label

\begin{equation}
    E=mc^2 \label{eq_1}
\end{equation}

        \begin{equation}
            F_m(x) = F_{m-1}(x) + \gamma_m h_m(x)
        \end{equation}

Example of two equations \eqref{eq_2} and \eqref{eq_3} each with separate equation numbers and separate reference labels.

\begin{gather}
    E=mc^2 \label{eq_2}\\
    (a+b)^2 = a^2 + 2ab + b^2 \label{eq_3}
\end{gather}


Example of two equations with a single equation number \eqref{eq_4} and a single reference label

\begin{equation}
    \begin{aligned}
    \dot{A} &  =2i\alpha(t)B\\
    \dot{B} &  =2i\gamma(t)A
    \end{aligned}
    \label{eq_4}
\end{equation}


The State of the Art is usually the more extensive part of a research proposal, so it will expectedly develop over various paragraphs and sub-paragraphs. It should be accompanied by comprehensive references, which you list at the end of the proposal. Ideally, all influential books, book chapters, papers and other texts produced in the knowledge domain you are exploring which are of importance for your work should be mentioned here and listed at the end of the proposal. You should follow very strictly the appropriate referencing conventions and make sure that no document you refer to is missing in the final list of references, nor vice versa. The choice of referencing conventions may depend on the specific field where your research is located. Popular conventions are those established by the \ac{IEEE}, \ac{APA}, \ac{ACM}, \ac{AIS} etc.


\section{First paragraph}

This is the first paragraph.

\subsection{First sub-paragraph of first paragraph}

As the State of the Art is likely to extend for some pages, it may need to be split into various paragraphs, with appropriate titles, and these paragraphs may need to be broken up further into sub-paragraphs. The paragraphs and sub-paragraphs should comply with the format used here.

\subsection{Second sub-paragraph of first paragraph}

This is an example of the second sub-paragraph of the first paragraph of the introduction.

\section{Second paragraph}

This is the second Paragraph

\subsection{First sub-paragraph of second paragraph}
This is an example of the first sub-paragraph of the second paragraph of the introduction.

\subsubsection{Second sub-sub-paragraph of second sub paragraph}
This is an example of the second sub-paragraph of the second paragraph of the introduction.