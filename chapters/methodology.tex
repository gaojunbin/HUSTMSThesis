The chapter Research Objectives and Approach clarifies the research objectives of your project, taking as its background your description of the state of the art, and describes the methodological approaches you have in mind to face the key research challenges of your project. The clarification of the research objectives should build solidly on the State of the Art and relate your research to the work carried out by others. It should elucidate the measure to which your work develops from their work and the extent to which it diverges from theirs to open up new and yet unexplored avenues. In essence, the chapter Research Objectives and Approach explains what you plan to do to tackle your research problem, why you plan to do it that way, and how you are going to do it.


\begin{figure}[b]

  \includegraphics[width=0.3\textwidth]{figures/example_1.jpg}
  \caption{\label{fig_1}Example of single figure - All figures position is center and top by default - to change the figure position please use position command after begin{figure} - \href{ https://www.overleaf.com/learn/latex/Positioning_of_Figures}{Click Here to see further}}
  
\end{figure}


\begin{algorithm}
  \caption{\label{alg_1}Example of Single Algorithm - Algorithm position is center and bottom by default - to change the figure position please use position command after begin{algorithm} - \href{https://www.overleaf.com/learn/latex/algorithms}{Click Here to see further}}
  
  \begin{algorithmic}[1]
    
    \Procedure{Euclid}{$a,b$}\Comment{The g.c.d. of a and b}
    \State $r\gets a\bmod b$
    \While{$r\not=0$}\Comment{We have the answer if r is 0}
    \State $a\gets b$
    \State $b\gets r$
    \State $r\gets a\bmod b$
    \EndWhile\label{euclidendwhile}
    \State \textbf{return} $b$\Comment{The gcd is b}
    \EndProcedure
  
  \end{algorithmic}

\end{algorithm}


The “how to” component of the proposal is called the Research Methods, or Methodology, component. It should be detailed enough to let the reader decide whether the methods you intend to use are adequate for the research at hand. It should go beyond the mere listing of research tasks, by asserting why you assume that the methods or methodologies you have chosen represent the best available approaches for your project. This means that you should include a discussion of possible alternatives and credible explanations of why your approach is the most valid. \textbf{Fig \ref{fig_1}} shows the example of single figure, \textbf{fig \ref{fig_2}} shows the example of 3-figures Subfloating and \textbf{fig \ref{fig_3}} shows the example of 2-figures Subfloating.





\begin{figure}

    \subfloat[Caption(a)]{\includegraphics[width=0.33\textwidth]{figures/example_1.jpg}\label{ex_1}}
    \subfloat[Caption(b)]{\includegraphics[width=0.33\textwidth]{figures/example_2.jpg}\label{ex_2}}
    \subfloat[Caption(c)]{\includegraphics[width=0.33\textwidth]{figures/example_3.jpg}\label{ex_3}}
    \caption{\label{fig_2}Example of 3-figures Subfloating - All figures position is center and top by default - to change the figure position please use position command after begin{figure} - \href{ https://www.overleaf.com/learn/latex/Positioning_of_Figures}{Click Here to see further}}
    
\end{figure}


\begin{figure}

    \subfloat[Caption(a)]{\includegraphics[width=0.4\textwidth]{figures/example_1.jpg}\label{ex_11}}
    \subfloat[Caption(b)]{\includegraphics[width=0.4\textwidth]{figures/example_2.jpg}\label{ex_22}}
    \caption{\label{fig_3}Example of 2-figures Subfloating - All figures position is center and top by default - to change the figure position please use position command after begin{figure} - \href{ https://www.overleaf.com/learn/latex/Positioning_of_Figures}{Click Here to see further}}
    
\end{figure}